\documentclass[11pt]{report}
\makeatletter\def\input@path{
{./Figures/}{./Files/}
}
\usepackage[utf8]{inputenc}
\usepackage[T1]{fontenc}
\usepackage{amsmath}
\usepackage{amsfonts}
\usepackage{amssymb}
\usepackage{graphicx}
\usepackage{subcaption}
\usepackage{xcolor}  
\usepackage[french]{babel}
\usepackage{array}
\usepackage{booktabs}
\usepackage{ragged2e}
\usepackage{siunitx}
\usepackage{tabularx} 
\usepackage{caption}
              % pour \SI{}, \celsius, etc.
\sisetup{detect-all}
\usepackage[version=4]{mhchem}     % pour \ce{^1O2}, \ce{O2}
              % pour \toprule, \midrule, \bottomrule
             % pour tabularx
\usepackage{float}                 % pour [H]

              % si tu utilises \textcolor{}

\usepackage{geometry}
% CHANGEMENT IMPORTANT : Utiliser backend=bibtex au lieu de biber
\usepackage[backend=bibtex, style=numeric, citestyle=numeric, sorting=none]{biblatex}
\addbibresource{PDT_PTT.bib}  % Votre fichier .bib
\usepackage[
    colorlinks=true,      % Active les liens colorés
    linkcolor=black,      % Couleur des liens internes (sections, etc.)
    citecolor=blue,       % Couleur des citations (vos références)
    urlcolor=blue,        % Couleur des URLs
    filecolor=blue        % Couleur des liens fichiers
]{hyperref}

\geometry{
 a4paper,
 total={170mm,257mm},
 left=20mm,
 top=20mm,
}

\setlength{\parindent}{0pt}
\setlength{\parskip}{1ex plus 0.5ex minus 0.2ex}

\newcommand{\suggestedfigure}[1]{\noindent \textbf{Suggestion de Figure :} \textit{#1}}
\title{\Large{Stage de Master} \\ Element de la revue bibliographique \\ \textit{\small Rapport de Laboratoire}}




\author{Corine KENGNE} 
%\address{Department of Physics \\ Faculty of Science, University of Yaounde 1}
\begin{document}
\maketitle
%\input{testgrok}
\chapter{Cancer du sein triple négatif}

%TODO qjouter une introduction avec annonce du plan
\section*{Introduction}

Le \textbf{cancer du sein triple négatif (TNBC)} constitue un défi thérapeutique majeur en raison de son agressivité, de sa chimiorésistance et de l’absence de cibles moléculaires classiques. Ce chapitre présente d’abord les aspects généraux du cancer du sein, avant de se focaliser sur le TNBC : définition moléculaire, épidémiologie, caractéristiques cliniques, vulnérabilités biologiques exploitables et justifications du choix de ce modèle pour le développement d’agents théranostiques combinant thérapie photodynamique (PDT) et photothermique (PTT) à base de nanoparticules de BODIPY.


\section{Cancer du sein: aspects généraux}\label{sec:TBNC_g }

Le \textbf{cancer} est une pathologie caractérisée par une prolifération cellulaire anarchique et une capacité à envahir les tissus voisins  \cite{Hanahan2011, Das2023, Ma2025}. 


\subsection{ Définition et physiopathologie du cancer du sein}

Le \textbf{cancer du sein } est une pathologie d'une prévalence alarmante à l'échelle mondiale, constituant la maladie la plus fréquemment diagnostiquée chez la femme \cite{Schaefer2025, Liu2025}. En 2022, les statistiques de l'Organisation Mondiale de la Santé (OMS) indiquaient que le cancer du sein avait été diagnostiqué chez \textbf{2,3 millions de femmes} dans le monde, entraînant environ \textbf{670 000 décès} \cite{Ailioaie2025}. Elle représente la première cause de mortalité par cancer chez la femme dans le monde \cite{Ailioaie2025} et sa complexité  réside dans sa nature non maîtrisée de prolifération cellulaire, sa forte tendance aux métastases, et sa résistance aux protocoles thérapeutiques standard \cite{Kumar2025}

\begin{figure}[htbp]
    \centering
    \includegraphics[width=0.5\linewidth]{sein_tumeur_anatomie.png}
    \caption{\textcolor{blue}{Représentation schématique d’un sein atteint d’une tumeur maligne}. La tumeur (en rouge) se développe à partir des canaux ou lobules mammaires et peut envahir les tissus adjacents (adapté de \protect\cite{Pinho2024})}
    \label{fig:sein-tumeur}
\end{figure}

Le processus de carcinogenèse commence par des mutations qui permettent aux cellules de rompre les régulations normales \cite{Beheshti2022}.La mutation initiale permet aux cellules de se multiplier sans contrôle \cite{Das2023, Marte2004}.
\begin{figure}[htbp]
    \centering
    \includegraphics[width=0.5\linewidth]{progression _tumeur.png}
    \caption{Progression tumorale et stadification clinique (T1 à T4) du cancer du sein selon la taille et l’envahissement locorégional.\textcolor{blue}{ Source : American Cancer Society (adapté)}}
    \label{fig:cancersein}
\end{figure}

• \textbf{division incontrôlée} : Le cancer est défini comme une condition impliquant une division et une prolifération cellulaire incontrôlées \cite{Das2023, Marte2004, Lee2024,Hayata2020}.

• \textbf{résistance à la mort cellulaire} : Les néoplasmes sont caractérisés par l'acquisition d'une résistance à la mort cellulaire et par le maintien de la signalisation proliférative \cite{Hanahan2011, Ailioaie2025, Ailioaie2025a}.

• \textbf{formation de tumeur} : Dans un premier temps, les cellules cancéreuses croissent et se multiplient de manière incontrôlable, se répandant dans les tissus environnants pour former un nodule tumoral \cite{Pinho2024} 
\begin{figure}[htbp]
    \centering
    
        \includegraphics[width=\textwidth]{progression _tumeur.png}
               \label{fig:image1}
    
    \caption{\textcolor{blue}{Formaition et évolution de la tumeur}}
    \label{fig:combined}
\end{figure}

Le cancer du sein  est la malignité la plus fréquente chez la femme dans le monde \cite{Schaefer2025}. En raison de sa complexité biologique, il est impératif de le classer en différents sous-types, ce qui guide le pronostic et la stratégie thérapeutique \cite{Ma2025, Go2025}.

\subsection{  Classification des cancers du sein}

La classification du cancer du sein est basée sur des critères combinant l'histopathologie et les caractéristiques moléculaires (ou statut des récepteurs) \cite{Go2025, Paulson2025}.

\begin{itemize}
    \item La classification histologique est basée sur l'arrangement microscopique, l'architecture et le patron des glandes \cite{Ailioaie2025, Ailioaie2025a}:
    
            \begin{figure}[htbp]
                \centering
                 \includegraphics[width=0.5\linewidth]{Typesein.png}
                  \caption{\textcolor{blue}{Schéma anatomique du sein.} localisation des principaux types de cancers mammaires}
                   \label{fig:classification_moléculaire}
                    \small Source : \cite{Go2025, Paulson2025}
            \end{figure}   

 

 \item  \textbf{La classification moléculaire selon l'expression des récepteurs}: 
 \begin{table}[h]
\centering
\caption{\textcolor{blue}{Classification moléculaire des cancers du sein}}
{\small
\begin{tabular}{|>{\RaggedRight}m{4cm}|>{\RaggedRight}m{8cm}|}
\toprule
\textbf{Type de cancer du sein} & \textbf{Profil de récepteurs} \\
\midrule
Cancer du sein Luminal A & Présence d'expression des récepteurs aux œstrogènes ($ER^+$) et à la progestérone ($PR^+$), mais absence d'expression du récepteur HER2 ($HER2^-$). \\
\midrule
Cancer du sein triple négatif & Absence ou faible expression des trois récepteurs : $ER^+$, $PR^+$, $HER2^-$. \\
\midrule
Cancer du sein Luminal B & Expression des récepteurs aux œstrogènes ($ER^+$) et/ou à la progestérone ($PR^+$), avec possible expression de HER2 ($HER2^+$) ou forte prolifération  \\
\midrule
Ciblage nanotechnologique & Manque de cibles intrinsèques pour la thérapie conventionnelle, développement de stratégies comme l'utilisation de molécules cationiques pour cibler les mitochondries ou de thérapies combinées (PDT/PTT). \\
\bottomrule
\end{tabular}}
 \small Source : \cite{Li2022, Ailioaie2025}
\end{table}


    
\end{itemize}

\section{ Le cancer du sein triple négatif }

\subsection{Définition et caractérisation moléculaire}

Le \textbf{TNBC (Triple Negative Breast Cancer)} est un sous-type de cancer du sein défini par \textit{l’absence ou la très faible expression des récepteurs aux œstrogènes (ER), à la progestérone (PR) et du récepteur 2 du Facteur de Croissance Épidermique Humain (HER2)} \cite{Li2022, Sun2020,Ma2025,James2024}.
Il représente environ \textbf{10 à 20\%} des cas de cancer du sein \cite{Syeda2025, Ndongwe2024, Chuang2025,Kumar2025a} ,  ou environ \textbf{10 à 15\%}  des cas invasifs \cite{Ailioaie2025}, et est caractérisé par l'absence ou la faible expression des trois récepteurs majeurs \cite{Ma2025,James2024,Ndongwe2024}:





\begin{figure}[htbp]
    \centering
    \begin{subfigure}[b]{0.5\textwidth}
        \centering
        \includegraphics[width=\textwidth]{TNBC.jpeg}
       
        \label{fig:image1}
    \end{subfigure}
   %( \hfill
     %\begin{subfigure}[b]{0.3\textwidth}
       %  \centering
         %\includegraphics[width=\textwidth]{absence.png}
       
         %\label{fig:image2}
    % \end{subfigure}
    \caption{\textcolor{blue}{Schéma comparatif d’une cellule de cancer du sein luminal (gauche) et d’une cellule TNBC (droite).} montrant l’absence des récepteurs ER, PR et HER2 à la surface membranaire}
    \label{fig:tnbc-cellule}
\end{figure}
%Profil Moléculaire du TNBC: Représentation d'une cellule TNBC montrant l'absence des récepteurs ER, PR et HER2
\subsection{Causes et distinction par rapport à d'autres cancers}

\subsubsection{Causes du TNBC}
Le développement et la progression du TNBC sont fortement influencés par \cite{Lebedeva2025}:

\textbf{Altérations Épigénétiques :} ces mécanismes comprennent la méthylation de l'ADN, les modifications des histones et le silençage génique médié par l'ARN non-codant (ncRNA)  \cite{Go2025}.

\textbf{Signatures Microbiologiques :} une analyse méta-transcriptomique révèle des signatures microbiennes et transcriptomiques distinctes dans les tumeurs TNBC basées sur l'ascendance raciale. Par exemple, les tumeurs AA présentent une abondance plus élevée de genres bactériens comme Hafnia et Cedecea \cite{Kumar2025a}.

\textbf{Dysrégulation des voies de signalisation :} De nombreuses voies de signalisation sont exprimées de manière aberrante dans le TNBC. Le dysfonctionnement de ces voies, ainsi que l'activité de molécules comme miR-301a-3p qui supprime PTEN (un suppresseur de tumeur) dans les cellules tumorales et le microenvironnement, favorise sa progression  \cite{Lin2025}.

\subsubsection{Distinction par rapport à d'autres cancers}
Le TNBC se distingue par l'absence de cibles conventionnelles, mais aussi par ses faiblesses exploitables, ce qui en fait un excellent modèle pour la nanomédecine \cite{Ma2025, Chuang2025}:

\textbf{Résistance aux thérapies ciblées :} L'absence des récepteurs ER, PR et HER2 signifie que le TNBC est insensible aux traitements anti-hormonaux et anti-HER2  \cite{Ma2025, Paulson2025}. Ceci est la distinction la plus nette, par exemple, au cancer du sein luminal A (ER+, PR+, HER2-), qui possède une vulnérabilité et une stratégie thérapeutique établies via la voie hormonale \cite{Lebedeva2025}.

\textbf{Dépendance à la chimiothérapie :} Le traitement du TNBC repose traditionnellement sur la chimiothérapie (seule ou combinée avec l'immunothérapie), mais cette approche est souvent compromise par la chimiorésistance et la toxicité systémique de ce dernier car il est l'archétype des cancers chimio-résistants \cite{Ma2025, Gao2024}.

\subsection{Caractéristiques Cliniques et Pronostic}
Le TNBC est considéré comme le sous-type moléculaire \textbf{le plus agressif} et sa manifestation clinique est souvent liée à des dynamiques tumorales rapides et défavorables \cite{Ailioaie2025, Go2025, Ma2025, Sun2020} :

\textbf{Croissance et invasion :} il est très invasif, avec un taux de croissance plus rapide  \cite{James2024, Oehler2024};

\textbf{Métastases et récidive :} il présente un risque plus élevé de métastases et un taux de récidive élevé ce qui l'associe à un pronostic défavorable  \cite{James2024, Oehler2024};

\textbf{Démographie :} il affecte majoritairement les femmes plus jeunes \cite{Syeda2025, Oehler2024} et montre des disparités raciales, avec une incidence plus élevée chez les femmes d'ascendance africaine (AA) par rapport à celles d'ascendance européenne (EA) \cite{Kumar2025a, Oehler2024}.

\textbf{Déficit de cibles :} l'absence des cibles (récepteurs) rend le TNBC insensible aux thérapies hormonales et aux traitements anti-HER2, qui sont efficaces sur d'autres sous-types (le cancer du sein luminal A) car ces derniers sont souvent utilisés comme biomarqueurs en oncologie mammaire \cite{Ma2025, Kong_2023, Paulson2025}.

\subsection{Le Défi Thérapeutique}
L'absence de cibles conventionnelles rend le TNBC \textbf{insensible aux thérapies hormonales} et aux traitements ciblés \textbf{anti-HER2} \cite{Ma2025,Kong_2023,Paulson2025}, car ces récepteurs sont couramment utilisés pour la classification et le traitement du cancer du sein, ce qui en fait un défi thérapeutique majeur \cite{Daester2016}.
Le choix du TNBC est stratégique car il est considéré comme l'archétype des cancers chimio-résistants et est le sous-type moléculaire le plus agressif. L'absence de cibles conventionnelles distingue le TNBC du Cancer du Sein Luminal A (ER+, PR+, HER2-) \cite{Wang2025a, Liu2025}. Cette caractéristique en fait l'\textbf{archétype des cancers chimio-résistants}  \cite{Ma2025, Gao2024}.

\begin{table}[h]
\centering
\caption{\textcolor{blue}{Défis Thérapeutiques}}
\small
\begin{tabular}{|>{\raggedright\arraybackslash}m{4cm}|>{\raggedright\arraybackslash}m{8cm}|}
\hline
\textbf{Caractéristique du TNBC} & \textbf{Conséquence \& Distinction} \\
\hline
\textbf{Absence de cibles (ER/PR/HER2)} & Rend le TNBC \textbf{insensible} aux thérapies hormonales et aux thérapies anti-HER2 ciblées, contrairement au cancer du sein luminal A (ER+, PR+, HER2-)  \cite{Ma2025, Paulson2025} \\
\hline
\textbf{Chimiorésistance} & Le traitement standard repose sur la chimiothérapie, mais le TNBC est l'archétype des cancers chimio-résistants \cite{Kong_2023, Paulson2025, Lebedeva2025} \\
\hline
\textbf{Vulnérabilités exploitables} & Ce cancer est une cible pour les nanothérapies car il sur-exprime certains récepteurs (ex. : Sortilin 1 ou SORT1) et est sensible à la photothérapie, et présente un Microenvironnement Tumoral (TME) hypoxique et acide \cite{Ailioaie2025, Duan2024,Demeule2021} \\
\hline
\textbf{Ciblage Nanotechnologique} & Le manque de cibles intrinsèques pour la thérapie conventionnelle pousse au développement de stratégies de ciblage innovantes, comme l'utilisation de molécules cationiques pour cibler les mitochondries, ou de thérapies combinées (PDT/PTT)  \cite{WasifBaig2025, Cui2025}\\
\hline
\end{tabular}
\end{table}

\subsection{Les Stratégies de Ciblage et Vulnérabilités Biologiques}

\subsubsection{Ciblage Subcellulaire : La Mitochondrie}

Les cellules TNBC, en tant que cancers agressifs, présentent un métabolisme exacerbé:

\textbf{Ciblage Mitochondrial :} En raison de leur croissance effrénée, les cellules TNBC sont hyper-dépendantes de leurs mitochondries, car elles sont au c\oe{}ur du métabolisme énergétique et sont impliquées dans l'induction de l'apoptose  \cite{Ailioaie2025, Lee2024}. Certaines cellules cancéreuses, y compris le TNBC, exploitent le potentiel de membrane mitochondriale négatif pour attirer les molécules chargées positivement (cationiques). Ceci entraînant l'utilisation de groupements cationiques lipophiles comme stratégie d'accumulation des colorants (comme les BODIPY) dans la matrice mitochondriale du TNBC (par exemple, la lignée cellulaire MDA-MB-231)\cite{Ahmad2025, Bongo2025, Badon2023}.

\textbf{Protéine TSPO :} La protéine translocatrice (TSPO), située sur la membrane externe des mitochondries, est considérée comme un biomarqueur tumoral dont l'expression est corrélée à l'agressivité de la tumeur.

\subsubsection{Le Microenvironnement Tumoral (TME) comme Cible}
Le TME du TNBC est hostile et limite l'efficacité des thérapies classiques. Il est exploité par les nanovecteurs pour une activation sélective \cite{Ailioaie2025}:

\textbf{Hypoxie et Faiblesse en Oxygène :} le TNBC est un modèle où l'hypoxie entrave l'efficacité de la Thérapie Photodynamique (PDT) conventionnelle (Type II). Les stratégies innovantes impliquent des nanoplatformes qui atténuent l'hypoxie en délivrant de l'oxygène \cite{Ailioaie2025, Duan2024}.

\textbf{pH Acide :} le TME est en général légèrement acide (pH 6,5–7,2), cette acidité est due à l'activité glycolytique intense des cellules cancéreuses et elle est une opportunité pour concevoir des nanoparticules qui ne s'activent qu'au site tumoral\cite{Xu2025, Chao2023}.

\textbf{Glutathion (GSH) :} Les concentrations élevées de glutathion dans le TME peuvent dégrader les espèces réactives de l'oxygène (ROS) \cite{Li2025b, Ao2025}. La déplétion du GSH est une stratégie nanotechnologique avérée pour améliorer l'efficacité des photothérapies contre le TNBC \cite{Ao2025}.

\textbf{Cibles de surface spécifiques :} Certains récepteurs sont surexprimés, malgré l'absence d'ER/PR/HER2. Par exemple, le récepteur SORT1 est surexprimé par le TNBC, et peut être ciblé par des conjugués peptide-médicament (PDCs) comme TH1902 \cite{Chuang2025}.

\subsubsection{ Justification du modèle TNBC les Thérapies Combinées}
Étant donné l'agressivité et la résistance du TNBC, les stratégies thérapeutiques modernes s'orientent vers des approches combinées multimodales pour attaquer le cancer sur plusieurs fronts \cite{Ailioaie2025, Li2025, Kong_2023}.
La recherche actuelle pour le TNBC se tourne vers des \textbf{nanoparticules (NPs)} et des systèmes d'administration innovants qui capitalisent sur les vulnérabilités identifiées du cancer  \cite{Zhu2024}.

\begin{itemize}
    \item \textbf{Thérapie Combinée :} La combinaison de la PDT (génération de ROS) et de la PTT (génération de chaleur) dans une seule nanoplateforme est une approche synergique puissante, notamment pour le TNBC  \cite{Ma2021, Ailioaie2025}.
    \item \textbf{Lutte contre la Thermorésistance :} Des systèmes thermo-répondants basés sur des peptides (ELP) et une hyperthermie douce ($41^{\circ}C$) ont été testés sur des cellules TNBC (MDA-MB-231) pour améliorer l'accumulation des médicaments et inhibir la signalisation Notch, réduisant potentiellement la chimiorésistance et la thermorésistance \cite{Go2025}.
    \item \textbf{Ferroptose :} Le TNBC est sensible à l'induction de la ferroptose, une forme de mort cellulaire régulée, dépendante du fer et de la peroxydation lipidique \cite{Zhu2024}. Des nanoparticules sont conçues pour interférer avec le métabolisme du fer et du GSH, pouvant être potentialisées par l'approche PTT/PDT \cite{Li2025}.
\end{itemize}
\section*{conclusion}
Le TNBC, par son absence de cibles conventionnelles, son agressivité et ses vulnérabilités métaboliques bien caractérisées, représente un modèle idéal pour développer des agents théranostiques innovants combinant imagerie par fluorescence et thérapie photodynamique/photothermique activée dans la fenêtre NIR-I. La conception de BODIPY modifiés (iodation positions 2,6, substitution 3,5 par donneurs $\pi$, greffage de vecteurs mitochondriaux cationiques) vise précisément à transformer ces faiblesses en opportunités thérapeutiques sélectives \cite{Zhang2022, Wang2025c, Lu2025, Overchuk2023, Li2025}.

 \chapter{Thérapie photodynamique (PDT), photothermique (PTT) et synergie PDT/PTT}
\label{chap:PDT-PTT}
Les recherches menées sur le traitement du cancer ont connue une évolution rapide réduisant le taux de mortalités et améliorant les conditions de vies de millions de patients. Contrairement aux méthodes
classiques de traitement telles que la radiothérapie et la chimiothérapie qui ont des effets secondaires et des récidives , certaines approches thérapeutiques innovantes basées sur la lumière et des photosensibilisateurs ont été un véritable domaine de recherche visant à détruire de manière complète les tissus tumoraux.



\section{Thérapie photodynamique (PDT)}

\subsection{Définition et Principe}

La \textbf{PDT} est une approche thérapeutique \textit{innovante, accessible et non-invasive} qui repose sur l'activation de molécules spécifiques par la lumière \cite{Ferroni2019, Correia2021, Kwiatkowski2018, Sun2019}.


C'est un traitement qui utilise la \textbf{lumière}, un \textbf{photosensibilisateur (PS)} et l'\textbf{oxygène moléculaire ($O{_2}$)} pour cibler et éliminer sélectivement les cellules cancéreuses \cite{Sun2019, Bongo2025, Ning2025}. Elle est reconnue pour sa \textbf{grande précision spatio-temporelle} et son \textbf{caractère minimalement invasif} \cite{Ailioaie2025, Lee2024, Ferroni2019}.



\subsection{Mécanisme d'action}
Le mécanisme se déroule en plusieurs étapes photophysiques et photochimiques après l'\textbf{administration du PS} :
\begin{itemize}
    \item \textbf{ Excitation par la lumière} : Le PS, localisé dans le tissu pathologique, absorbe un photon (généralement dans la fenêtre thérapeutique (600–850 nm)) et passe de son état singulet fondamental ($S{_0}$) à un état singulet excité ($S{_1}$) \cite{Lee2024, Ning2025, Kwiatkowski2018, Hilf_2020}.

    \item \textbf{ Transition Inter-Système (ISC)} : une fois le PS excité, il subit ensuite une transition sans rayonnement vers l'état triplet excité ($T{_1}$) \cite{Kwiatkowski2018, Ning2025, Lee2024}. Ce passage est facilité par l'effet d'atome lourd (si des halogènes sont incorporés au PS) \cite{Bongo2025, Sun2019}.
    
     \item \textbf{Production d'espèces réactives de l'oxygène (ROS)} : L'état triplet $T{_1}$ peut réagir via deux mécanismes principaux \cite{Kwiatkowski2018, Lee2024} :

     \begin{itemize}
         \item \textbf{Réaction de Type II (la plus courante)} : le PS transfère son énergie à l'\textit{oxygène moléculaire de l'environnement (${^3}O{_2}$)}, générant de l'\textbf{oxygène singulet (${^1}O{_2}$)}, qui est la principale espèce cytotoxique \cite{Kwiatkowski2018, DosSantos2019, Ning2025, Lee2024}.
        
         \item \textbf {Réaction de Type I }: le PS transfère un \textit {électron ou un atome d'hydrogène à des substrats adjacents}, formant d'autres ROS comme le \textbf{radical hydroxyle (HO)} ou le \textbf{peroxyde d'hydrogène ($H{_2}O{_2}$)} \cite{Kwiatkowski2018, DosSantos2019, Lee2024}.
        \end{itemize} 
 
    \item \textbf{Mort cellulaire} : les ROS ainsi générées induisent un stress oxydatif qui endommage les membranes, les protéines, les lipides et l'ADN/ARN, entraînant la mort cellulaire par \textbf{apoptose, nécrose ou autophagie} \cite{DosSantos2019, Donohoe2019, Lee2024, Przygoda2023, Hilf_2020}.
   
\end{itemize}

\begin{figure}[htbp]
    \centering
    \includegraphics[width=0.85\linewidth]{Figures/PDT_mecanisme_Jablonski.png}
    \caption{\textbf{Diagramme de Jablonski illustrant les processus photophysiques de la PDT} : absorption d’un photon, transition inter-système (ISC favorisée par effet atome lourd), réactions de type I (transfert d’électron) et type II (transfert d’énergie vers $\ce{^3}O{_2}$ → $\ce{^1}O{_2}$) \cite{Zhao2024,He2025}.}
    \label{fig:jablonski-pdt}
\end{figure}


\subsection{Avantages et Inconvénients}

\begin{table}[htbp]
\centering
\caption{\textbf{Avantages et limitations de la thérapie photodynamique (PDT) en oncologie}. Ce tableau synthétise les principaux bénéfices thérapeutiques de la PDT ainsi que les défis techniques et biologiques qui limitent son application clinique, selon la littérature récente.}
\label{tab:avantages_limitations_pdt}
\begin{tabular}{|p{0.45\textwidth}|p{0.45\textwidth}|}
\hline
\textbf{Avantages}  & \textbf{Inconvénients (Limitations)}  \\
\hline
Sélectivité spatio-temporelle élevée \cite{ Bongo2025} & Dépendance à l'oxygène : Efficacité compromise dans le microenvironnement tumoral hypoxique (TME) \cite{Ailioaie2025, Denko2008, Kwiatkowski2018, Li2020} \\
\hline
Faible toxicité systémique et effets secondaires minimes \cite{Ailioaie2025} & Pénétration limitée de la lumière dans les tissus profonds \cite{Ailioaie2025, Li2020, Kong2022} \\
\hline
Faible probabilité de résistance tumorale \cite{Ferroni2019, Kong2022} & Photosensibilité cutanée prolongée due à une clairance lente du PS des tissus sains \cite{Ferreira-Goncalves2021, Park2021, Lee2024}\\
\hline
Capacité à stimuler des réponses immunitaires antitumorales \cite{Ailioaie2025, Lee2024, Kong2022} & 
Problèmes de solubilité et de photostabilité des PS conventionnels \cite{Lee2024, Ailioaie2025}\\
\hline
\end{tabular}
\end{table}

% Note : Assurez-vous d'inclure ces packages dans le préambule :
% \usepackage{array}
% \usepackage[table]{xcolor} % optionnel, pour des couleurs
% \usepackage{cite} % pour la gestion des citations


\section{Thérapie photothermique (PTT)}

\subsection{Définition et Principe}

La \textbf{thérapie photothermique (PTT)} est une stratégie thérapeutique \textit{prometteuse, simple et peu invasive} qui utilise la conversion de la lumière en chaleur pour détruire les cellules \cite{Ferroni2019, Zhao2021, Lopes2021}.

Elle implique l'utilisation d'\textit{agents photothermiques (PTAs)} qui absorbent l'énergie lumineuse et la convertissent en énergie thermique, provoquant une hyperthermie (chaleur locale supérieur $\SI{41}{\celsius}$) localisée pour détruire les cellules cancéreuses \cite{Zhao2021, Ferroni2019} par \textbf{dénaturation protéique et rupture membranaire} \cite{Chao2023,Skrodzki2024}. .


\subsection{Mécanisme d'action}
Le mécanisme est basé sur la désactivation non-radiative de l'état excité \cite{Han2021, Sun2019} :

\begin{itemize}
    \item 

\textbf{ Absorption }: Le PTA absorbe les photons, le faisant passer à un état singulet excité \cite{Han2021}.

    \item \textbf{Conversion Non-Radiative }: au lieu d'émettre de la fluorescence ou de subir un ISC, l'énergie de l'état excité se dissipe sous forme de vibrations moléculaires par relaxation non-radiative en revenant à l'état fondamental \cite{Han2021, Sun2019}. Cette dissipation se traduit par une \textbf{augmentation de la température} dans le PTA et son microenvironnement \cite{Zhao2021}.
    
    \item  \textbf{Hyperthermie et destruction cellulaire }: lorsque la température locale dépasse \SI{41}{\celsius}, l'hyperthermie provoque la dénaturation des protéines cellulaires et l'agrégation, conduisant à la destruction des cellules tumorales \cite{Kumari2021, Melamed2015}.

    \begin{itemize}
        \item  Des températures modérées (\SIrange{41}{47}{\celsius}) induisent \textit{l'apoptose} \cite{Melamed2015, PerezHernandez2018}.
        \item Des températures élevées (superieur $\SI{55}{\celsius}$) peuvent entraîner la \textit{nécrose} \cite{Melamed2015, PerezHernandez2018}.
    \end{itemize} 

\end{itemize}

\begin{figure}[htbp]
    \centering
    \includegraphics[width=0.75\linewidth]{Photothermal.png}
    \caption{\textcolor{blue}{Schéma de la thérapie photothermique (PTT) appliquée au cancer du sein.} Des nanoparticules multifonctionnelles (ex. AuNPs ou BODIPY iodés) s’accumulent sélectivement dans la tumeur grâce à l’effet EPR et à un éventuel ciblage actif. Sous irradiation NIR ($\SIrange{650}{900}{nm}$), elles convertissent l’énergie lumineuse en chaleur locale (> $\SI{42}{\celsius}$), provoquant une hyperthermie létale et la mort des cellules cancéreuses par nécrose/apoptose. La chaleur générée potentialise également la perfusion tumorale et l’oxygénation, bénéficiant à une éventuelle PDT combinée \cite{Yang2023,He2025,Zhao2024,Chao2023}}
    \label{fig:ptt-chaleur}
\end{figure}

Avantages clés :
\begin{table}[htbp]
    \centering
    \caption{Avantages et inconvénients de la thérapie photothermique (PTT) dans le contexte du cancer du sein triple négatif. La combinaison PTT/PDT à base de nanoparticules BODIPY permet de contourner la plupart des limitations listées ici (indépendance à l’oxygène, thermorésistance, délivrance suboptimale) tout en conservant les bénéfices d’une ablation thermique localisée \cite{Kong2022,Chao2023,Zhao2021,Lee2024}.}
    \label{tab:avantages-inconvenients-PTT}
    \begin{tabularx}{\textwidth}{>{\centering\arraybackslash}p{6.8cm}>{\centering\arraybackslash}p{6.8cm}}
        \toprule
        \textbf{Avantages de la PTT} & \textbf{Inconvénients / Limitations de la PTT} \\
        \midrule
        Indépendance complète vis-à-vis de l’oxygène : efficacité préservée dans le microenvironnement hypoxique typique du TNBC \cite{Eskiizmir2017,Kong2022,Lee2024} &
        Thermorésistance possible via surexpression des protéines de choc thermique (HSP70, HSP90) \cite{Xin2022,Cao2022,Deng2021} \\[8pt]
        
        Sélectivité spatio-temporelle élevée et ablation très localisée \cite{Kong2022} &
        Risque théorique de dissémination tumorale par dilatation vasculaire et augmentation du flux sanguin lors d’un chauffage prolongé ou mal contrôlé \cite{Chao2023} \\[8pt]
        
        La chaleur (> \SI{42}{\celsius}) sensibilise les cellules tumorales à la chimiothérapie et à l’immunothérapie \cite{Li2022} &
        Pénétration limitée de la lumière dans les tissus profonds (nécessite irradiation NIR-I ou NIR-II) \cite{Li2020,Han2021,Kong2022} \\[8pt]
        
        Faible incidence de résistance tumorale (comparable à la PDT) \cite{Ferroni2019} &
        Efficacité de délivrance parfois suboptimale des agents photothermiques (PTA) au site tumoral \cite{Ferroni2019} \\
        \bottomrule
    \end{tabularx}
\end{table}

\section{Synergie PDT/PTT : une stratégie optimale pour le TNBC}

La combinaison des deux modalités photothérapeutiques ( PDT/PTT)) est une stratégie multimodale puissante et hautement souhaitable pour obtenir des effets synergiques ("1+1>2") \cite{Sun2019, Liu2022, Chao2023, Grin2022}  dans une même nanoparticule \cite{Yang2023,Zhao2024,He2025,Chen2024}. La principale motivation de cette combinaison PDT/PTT est que chaque technique compense les \textit{faiblesses de l'autre} \cite{Sun2019, Kong2022, Ailioaie2025} :

\begin{table}[htbp]
    \centering
    \caption{Complémentarité PDT/PTT dans le microenvironnement tumoral du TNBC}
    \label{tab:synergie}
    \begin{tabularx}{\textwidth}{>{\raggedright\arraybackslash}X>{\raggedright\arraybackslash}X}
        \toprule
        \textbf{Limitation} & \textbf{Compensation par l’autre modalité} \\
        \midrule
        Hypoxie tumorale (PDT limitée):  Le défi majeur de la PDT (Type II) est l'environnement hypoxique de la tumeur \cite{Ailioaie2025} & PTT indépendante de l’\ce{O2} ce qui renforce l'efficacité de la PDT \cite{Chao2023,Yang2023,Sun2019, Kong2022, Lee2024} \\
        Thermorésistance : L'approche PTT seule peut échouer si les cellules restantes développent une résistance à la chaleur (HSP, PTT limitée) & La PDT génère des ROS qui tuent ces cellules thermorésistantes, garantissant une ablation tumorale plus complète \cite{Sun2019, Kong2022, Zhao2024} \\
        \bottomrule
    \end{tabularx}
\end{table}

L'intégration de la PDT et de la PTT se fait couramment via des nanoplateformes théranostiques \textit{(par exemple, des nano-photosensibilisateurs à base de BODIPY)} qui combinent les propriétés des deux agents dans un seul système \cite{Sun2019, Ning2025, Liu2022, Li2020}. Ces nanoplates-formes peuvent être activées par une seule irradiation NIR, bénéficiant de la pénétration accrue de ces longueurs d'onde (700 à 1350 nm) \cite{Li2020, Kadkhoda2022}.

\begin{figure}[htbp]
    \centering
    \includegraphics[width=0.95\linewidth]{SynergiePDT_PTT.png}
    \caption{Schéma de la synergie PDT/PTT dans une nanoparticule BODIPY iodée ciblant les mitochondries du TNBC : (i) ISC renforcé par iode → forte génération de $\ce{^1O 2}$ (PDT), (ii) relaxation non radiative → hyperthermie (PTT), (iii) la chaleur augmente l’oxygénation locale, potentialisant la PDT, et (iv) mort cellulaire complète (apoptose + nécrose). D’après \cite{He2025,Zhao2024,Chen2024}.}
    \label{fig:synergie-tnbc}
\end{figure}

\section*{Conclusion partielle}

La combinaison PDT/PTT dans une nanoparticule BODIPY optimisée représente une stratégie théranostique particulièrement adaptée au TNBC : elle exploite l’hypoxie et l’hyper-dépendance mitochondriale de ce cancer tout en contournant ses résistances grâce à une double action cytotoxique activée par une seule irradiation NIR \cite{Yang2023,He2025,Chen2024,Zhao2024}.

Le succès de cette approche repose sur la conception rationnelle de nanoparticules théranostiques à base de BODIPY (boron-dipyrrométhène). Par iodation en positions 2,6 (effet atome lourd), substitution $\pi$ extenseur en méso et positions 3,5, et greffage de vecteurs mitochondriaux cationiques \cite{Jana2024,He2025,Zhao2024, Chen2024, WasifBaig2025}.

Ainsi, l’optimisation computationnelle et expérimentale de ces nanoparticules BODIPY iodées représente une avancée majeure vers une thérapie combinée PDT/PTT hautement sélective et efficace contre le cancer du sein triple négatif \cite{Yang2023,He2025,Zhao2024,Chen2024,Jana2024,Ailioaie2025}.




% \chapter{Nanoparticules à base de BODIPY pour la photothérapie}


% \section{Définition et Propriétés du BODIPY de base}
% \subsection{Définition}
% Le \textbf{BODIPY} est un colorant organique dont le nom IUPAC (nom officiel) est \textbf{4,4-difluoro-4-bora-3a,4a-diaza-s-indacène} \cite{Kumar2025, Das2023, Bongo2025}.
% Le terme BODIPY lui-même est un acronyme dérivé de sa composition qui signifie \textbf{Boron DI Pyrromethene} \cite{Kumar2025}.

% C'est un fluorophore organique polyvalent qui a émergé comme une plateforme essentielle en imagerie cellulaire et dans les applications thérapeutiques \cite{Kumar2025, Das2023}. C'est une classe de colorants organiques synthétisée pour la première fois en 1968 \cite{Kumar2025, Bongo2025,  Das2023, Ahmad2025}.
% Le squelette BODIPY présente de multiples positions:\textbf{ $\alpha$ (1, 2, 6, 7), $\beta$ (3, 5) et meso (8)} où des substituants peuvent être ajoutés. La capacité à modifier chimiquement le cœur BODIPY (ajustabilité spectrale) est une propriété clé qui permet de fine-tuner (ajuster) ses \textbf{longueurs d'onde d'absorption et d'émission}, crucial pour l'adapter à différentes applications, notamment la photothérapie dans la fenêtre NIR \cite{Das2023,Kumar2025}.

% \begin{figure}[h]
%     \centering
%     \begin{minipage}{0.45\linewidth}
%         \centering
%         \includegraphics[width=\linewidth]{BODIPY-1.png}
%     \end{minipage}
%     \hfill
%     \begin{minipage}{0.45\linewidth}
%         \centering
%         \includegraphics[width=\linewidth]{Bodipy.png}
%     \end{minipage}
%     \caption{\textcolor{blue}{BODIPY de base}}
%     \label{fig:placeholder}
% \end{figure}
% \subsection{Structure, propriétés physico-chimiques et optiques}
% Le BODIPY possède une structure centrale bien définie, qui lui confère ses propriétés photophysiques uniques \cite{Kumar2025, Das2023, Bongo2025}. %(Sa structure est caractérisée par un squelette \textbf{tricyclique rigide et planaire}, incluant un \textbf{atome de bore} \cite{Bongo2025, Ahmad2025}.
% Sa structure de base est un cadre tricyclique \cite{Bongo2025} formé des éléments suivants :

%  • \textbf{Squelette Central et forme plane} : Le c\oe{}ur du BODIPY est constitué d'un hétérocycle aromatique à cinq chaînons contenant de l'\textbf{azote}) \cite{Das2023}. Sa conformation \textbf{rigide et planaire} favorise une \textbf{bonne absorption et émission de la lumière} \cite{Bongo2025,Ahmad2025}.
 
%  • \textbf{Deux Unités Pyrrole} : Le squelette est composé de deux unités pyrrole reliées par un \textbf{Pont Méthine}(un atome de carbone)\cite{Bongo2025} 
  
%  • \textbf{Centre Difluorure de Bore} : Le centre de la molécule est stabilisé par un atome de \textbf{bore (B)}, qui est chélaté aux atomes d'azote des deux pyrroles \cite{Bongo2025}. Cet atome de bore est lié à deux atomes de \textbf{fluor (F)}, ce qui contribue à la \textbf{fluorescence} du composé. Cette coordination forme le groupement \textbf{difluorure de bore (BF${_2})$} \cite{Kumar2025}et assure la stabilité de la structure \cite{Bongo2025}.
 
%  Le BODIPY possède une structure bien définie, qui lui confère ses propriétés photophysiques uniques \cite{Kumar2025, Das2023, Bongo2025}. C'est une molécule polyvalente reconnue pour ces propriétés exceptionnelles, qui la rendent précieuse en imagerie cellulaire, en thérapie et qui en font de lui un candidat idéal pour la conception de photosensibilisateurs (PSs) \cite{Kumar2025, Ahmad2025, Das2023}:

% • \textbf{Haute Stabilité et Rendement Quantique }: Il possède une excellente photostabilité et des rendements quantiques de fluorescence ($\Phi_{\mathrm{fluo}}$) élevés (souvent proches de 0,8) \cite{Kumar2025, Das2023, Bongo2025,Nykaenen2024}, permettant une imagerie de haute fidélité \cite{Kumar2025, Ma2025a, Das2023}.

% • \textbf{Absorption et Émission} : Il présente de \textbf{forts coefficients d'extinction molaire} et des \textbf{bandes d'absorption} et d'\textbf{émission nettes} dans les régions visible et proche infrarouge (NIR) du spectres électromagnétiques avec des pics bien définis\cite{Kumar2025, Das2023, Bongo2025}.

% • \textbf{Ajustabilité Spectrale} : La structure BODIPY peut être facilement modifiée chimiquement, permettant d'ajuster les propriétés photophysiques (tunability) et de décaler leur absorption vers la fenêtre thérapeutique du NIR-I (600-900 nm), ce qui est essentiel pour une pénétration en profondeur dans les tissus et pour la conception des dérivés BODIPY \cite{ Kumar2025, Das2023}.



% %(Formulation en Nanoparticules L'utilisation de BODIPY dans les systèmes biologiques est limitée par leur nature intrinsèquement hydrophobe, ce qui peut entraîner une extinction due à l'agrégation et une faible solubilité dans l'eau \cite{Lipophilic quaternary ammonium-functionalized BODIPY photosensitizers for mitochondrial-targeted photodynamic therapy and fluorescence cell imaging205, Boron Dipyrromethene Nano Photosensitizers for Anticancer Phototherapies.pdf}. Pour surmonter ce défi, les BODIPY sont couramment formulés en nanoparticules (NPs) ou encapsulés, ce qui améliore leur solubilité aqueuse, leur biocompatibilité et leur accumulation au site tumoral via l'effet EPR (Enhanced Permeability and Retention) \cite{Boron Dipyrromethene Nano Photosensitizers for Anticancer Phototherapies.pdf, BODIPY Dyes_A New Frontier in Cellular Imaging and Theragnostic Applications205}.


% Le BODIPY est un fluorophore flexible, constituant une plateforme de développement de médicaments fascinante, dont les propriétés peuvent être ajustées par des modifications structurelles appropriées \cite{Das2023, Chen2024}.

% \section{Dérivés Spécifiques de BODIPY}
% Les modifications chimiques du squelette BODIPY permettent de basculer l'utilisation de la molécule vers l'imagerie (fluorescence) ou dans les photothérapie (PDT/PTT), afin d'optimiser la production d'espèces cytotoxiques (ROS et chaleur) et d'assurer le ciblage.\cite{Kumar2025}.


% \subsection{Dérivés Halogénés : Iodo-BODIPY }
% L'\textbf{Iodo-BODIPY} est un photosensibilisateur (PS) conçu pour maximiser la production d'espèces réactives de l'oxygène (ROS), essentielles à l'efficacité de la PDT \cite{ Kumar2025}. Il est obtenu en substituant \textbf{deux atomes d'Iode} aux positions \textbf{ $\alpha$ ( 2, 6)} du cœur BODIPY \cite{
% Kumar2025, WasifBaig2025 } à l'aide de réactifs d'addition électrophile (NBS et N-iodosuccinimide (NIS), respectivement) \cite{ Bongo2025, WasifBaig2025}.


% \begin{figure}[H] \textcolor{blue}{}
%     \centering
%     \includegraphics[width=0.5\linewidth]{iodo-Bodipy.png}
%     \caption{\textcolor{blue}{Iodo-BODIPY}}
%     \label{fig:placeholder}
% \end{figure}

% Pour que la Thérapie Photodynamique (PDT) soit efficace, le PS doit passer de l'état excité singulet ($\mathrm{S}_1$) à l'état triplet ($\mathrm{T}_1$) via le \textbf{Passage Intersystème (ISC)} \cite{Pinho2024, Awuah2012}.

% \textbf{Effet d'Atome Lourd (HAE) :} L'incorporation d'atomes lourds, tels que l'\textbf{iode} (Iodo-BODIPY), notamment aux positions \textbf{2} et \textbf{6} du cœur BODIPY, est la stratégie la plus courante \cite{Bongo2025}. Cette halogénation \textbf{réduit la fluorescence} ($\Phi_{\mathrm{fluo}}$) tout en \textbf{augmentant significativement la génération d'oxygène singulet ($^1\mathrm{O}_2$) }\cite{Bongo2025, Nguyen2021}, grâce à l'augmentation du couplage spin-orbite (SOC) qui favorise l'\textbf{ Intersystème (ISC)} \cite{Ponte2018, Zhang2013}. C'est pour cette raison que l'iode est souvent préféré au brome. \cite{ Baig2025a , Bongo2025, Das2023}

% \subsection{ TPP-BODIPY (Triarylphosphonium-BODIPY)}
% Le \textbf{TPP-BODIPY} est un BODIPY fonctionnalisé par un groupement \textbf{triarylphosphonium $(TPP^{+})$}, dont l'objectif principal est d'assurer la localisation spécifique du photosensibilisateur dans les mitochondries des cellules cancéreuses \cite{ Chen2024, Ahmad2025}.
% Ce groupement \textbf{triarylphosphonium $(TPP^{+})$} fait partie des cations lipophiles délocalisés les plus couramment utilisés comme \textbf{ "GPS biologique"} pour le ciblage mitochondrial \cite{ Bongo2025,Ahmad2025, Lee2024, Chen2024}.
% Le ciblage mitochondrial est essentiel car les mitochondries sont des régulateurs multifacettes de la mort cellulaire \cite{Bock2020}.

%  • Les BODIPY fonctionnalisés au \textbf{ $(TPP^{+})$}  sont des outils puissants pour l'imagerie mitochondriale dans les cellules vivantes, grâce à la \textbf{haute stabilité} et aux \textbf{rendements quantiques de fluorescence} de la base BODIPY \cite{Kumar2025, Ahmad2025}. Ils permettent d'évaluer la fonction et la santé mitochondriales \cite{Bongo2025}.
 
% • La combinaison d'un groupement \textbf{cationique lipophile} (comme l'\textit{ammonium quaternaire, très similaire au TPP+} dans sa fonction de ciblage) et d'une \textbf{halogénation (Iodo)} a permis la synthèse de \textbf{BODIPY cationiques}  qui présentent un \textbf{faible niveau de toxicité en l'absence de lumière (toxicité sombre négligeable)} mais une \textbf{cytotoxicité significative} après activation lumineuse contre des lignées cellulaires cancéreuses  \cite{Bongo2025}.

% En résumé, l'\textbf{Iodo-BODIPY} est la partie du complexe qui donne le pouvoir destructeur (PDT) en générant efficacement des \textbf{ROS grâce à l'effet d'atome lourd}, tandis que le \textbf{TPP-BODIPY} est la partie qui donne la \textbf{précision chirurgicale (le ciblage)}, en s'assurant que ce pouvoir destructeur est concentré sur les mitochondries des cellules cancéreuses \cite{Bongo2025}.


% \section{Conception de Nanoparticules (NPs) de BODIPY}

% Bien que les BODIPY soient efficaces, ils sont souvent intrinsèquement hydrophobes et peuvent souffrir de l'effet d'extinction induite par l'agrégation  en milieu aqueux \cite{Rong2024}. Le passage à des nano-photosensibilisateurs (NPs) permet de contourner ces limitations \cite{Sun2019, Chen2025}.

% \begin{table}[h]
% \centering
% \caption{Avantages et Méthodes d'Encapsulation des BODIPY en Nanoparticules}
% \label{tab:nps_bodipy}
% \begin{tabularx}{\linewidth}{|>{\raggedright\arraybackslash}p{6cm}|>{\raggedright\arraybackslash}X|}
% \hline
% \textbf{Avantages des NPs de BODIPY} & \textbf{Méthodes de Préparation} \\
% \hline
% \textbf{Amélioration de la Solubilité} et de la biocompatibilité \cite{Sun2019, Rong2024}. & 
% \textbf{Précipitation Moléculaire/Nanoprécipitation :} Utilisée pour les conjugués amphiphiles, comme le BrBDP-2PTX, qui s'auto-assemblent en NPs stables \cite{Zhang2018, Sun2019}. \\
% \hline
% \textbf{Accumulation Améliorée} au niveau tumoral via l'effet de perméabilité et de rétention accrues (EPR) et le ciblage actif \cite{Sun2019, Janakiraman2025}. & 
% \textbf{Encapsulation Polymérique :} Utilisation de polymères ou de liposomes pour enrober les BODIPY hydrophobes \cite{Sun2019, Malacarne2024, Porolnik2024}. \\
% \hline
% \textbf{Fonctionnalisation :} La matrice NP peut intégrer d'autres agents (chimio, ADN) ou être modifiée par des ligands de ciblage \cite{Ndongwe2024, Chen2025}. & 
% \textbf{Interactions Supramoléculaires :} Utilisation de cyclodextrines (CD) ou d'autres interactions pour former des nanostructures \cite{Sun2019}. \\
% \hline
% \end{tabularx}
% \end{table}

% \section{Rôle dans le Traitement du Cancer (PDT/PTT) Synergique}

% Les nanoparticules de BODIPY sont des agents théranostiques puissants qui facilitent la destruction des cellules cancéreuses et la surveillance en temps réel \cite{Kumar2025, Sun2019}.


% \begin{itemize}
%     \item \textbf{Thérapie Synergique :} Les nanoparticules de BODIPY sont au centre des plateformes théranostiques combinant la \textbf{ PDT (génération de ROS cytotoxiques)} et la \textbf{PTT (conversion de lumière en chaleur)} \cite{Kumar2025, Sun2019, Wang2025c}.
    
%     \item \textbf{Ciblage Subcellulaire :} Le ciblage mitochondrial par les \textbf{dérivés cationiques (TPP-BODIPY)} est crucial. Cette accumulation assure que la génération de ROS (via Iodo-BODIPY activé) détruit précisément le centre énergétique de la cellule, induisant l'apoptose \cite{Bongo2025, Ahmad2025, Qi2019, Lee2024}.
    
%     \item \textbf{Surmonter la Résistance :} L'ingénierie des NPs permet de contourner la \textbf{chimiorésistance}, par exemple en co-délivrant des agents cytotoxiques (comme le PTX) et le PS \cite{Zhang2018}.
    
%     \item \textbf{Exemples d'Application :} Des dimères de BODIPY iodés encapsulés dans des liposomes chargés positivement (DOTAP:POPC) ont démontré une activité élevée contre des lignées cellulaires agressives, notamment le \textbf{Cancer du Sein Triple Négatif }\cite{Porolnik2024}.
% \end{itemize}

% \vspace{1cm}

% \begin{center}
% \begin{figure}[h]
%     \centering
%     \includegraphics[width=0.5\linewidth]{Nanoparticules.jpg}
%     \caption{\textcolor{blue}{\textbf{ Nanoparticule de BODIPY Théranostique}:
% Illustration d'une nanoparticule encapsulant un BODIPY fonctionnalisé (Iodo-BODIPY pour PDT et/ou PTT) et un groupement TPP$^+$ pour le ciblage mitochondrial. Montrer l'activation par lumière NIR, la génération de chaleur/ROS et l'accumulation sélective dans la mitochondrie.}}
%     \label{fig:placeholder}
% \end{figure}
% \end{center}

% \newpage

% % Bibliographie manuelle
% \begin{thebibliography}{99}

% \bibitem{Kumar2025} Kumar, A. et al. (2025). \textit{BODIPY-based photosensitizers for cancer therapy}. Journal of Photochemistry.

% \bibitem{Das2023} Das, S. et al. (2023). \textit{BODIPY fluorophores in biomedical applications}. Chemical Reviews.

% \bibitem{Ma2025a} Ma, L. et al. (2025). \textit{Near-infrared BODIPY derivatives}. Advanced Materials.

% \bibitem{Sun2019} Sun, W. et al. (2019). \textit{BODIPY-based nanoparticles for phototherapy}. Nature Communications.

% \bibitem{Chen2024} Chen, Y. et al. (2024). \textit{Structural modifications of BODIPY}. Organic Letters.

% \bibitem{Pinho2024} Pinho, E. et al. (2024). \textit{Photodynamic therapy mechanisms}. Photochemical Sciences.

% \bibitem{Awuah2012} Awuah, S. G., You, Y. (2012). \textit{Boron dipyrromethene (BODIPY)-based photosensitizers for photodynamic therapy}. RSC Advances.

% \bibitem{Bongo2025} Bongo, M. et al. (2025). \textit{Iodo-BODIPY derivatives for enhanced PDT}. European Journal of Medicinal Chemistry.

% \bibitem{Nguyen2021} Nguyen, V. et al. (2021). \textit{Heavy atom effect in BODIPY photosensitizers}. Chemistry - A European Journal.

% \bibitem{Ponte2018} Ponte, F. et al. (2018). \textit{Spin-orbit coupling in halogenated BODIPYs}. Physical Chemistry Chemical Physics.

% \bibitem{Zhang2013} Zhang, X. F. et al. (2013). \textit{The mechanisms of photosensitization by halogenated BODIPY dyes}. Journal of Physical Chemistry A.

% \bibitem{Bock2020} Bock, F. J., Tait, S. W. G. (2020). \textit{Mitochondria as multifaceted regulators of cell death}. Nature Reviews Molecular Cell Biology.

% \bibitem{Qi2019} Qi, S. et al. (2019). \textit{Mitochondria-targeted photosensitizers for cancer therapy}. Chemical Society Reviews.

% \bibitem{Lee2024} Lee, H. et al. (2024). \textit{TPP-functionalized BODIPY for mitochondrial targeting}. Bioconjugate Chemistry.

% \bibitem{Rong2024} Rong, M. et al. (2024). \textit{Overcoming aggregation-caused quenching in BODIPY}. ACS Applied Materials \& Interfaces.

% \bibitem{Chen2025} Chen, J. et al. (2025). \textit{Nano-photosensitizers based on BODIPY}. Advanced Healthcare Materials.

% \bibitem{Zhang2018} Zhang, W. et al. (2018). \textit{BrBDP-2PTX nanoparticles for combined therapy}. Biomaterials.

% \bibitem{Janakiraman2025} Janakiraman, S. et al. (2025). \textit{EPR effect in tumor targeting}. Drug Delivery Reviews.

% \bibitem{Malacarne2024} Malacarne, D. et al. (2024). \textit{Polymeric encapsulation of photosensitizers}. Journal of Controlled Release.

% \bibitem{Porolnik2024} Porolnik, K. et al. (2024). \textit{Liposomal BODIPY dimers for triple-negative breast cancer}. Molecular Pharmaceutics.

% \bibitem{Ndongwe2024} Ndongwe, T. et al. (2024). \textit{Functionalized nanoparticles for targeted delivery}. Nanomedicine.

% \bibitem{Wang2025c} Wang, Z. et al. (2025). \textit{Synergistic PDT/PTT platforms}. Advanced Functional Materials.

% \bibitem{Ahmad2025} Ahmad, I. et al. (2025). \textit{ROS generation in mitochondria-targeted PDT}. Free Radical Biology and Medicine.

% \end{thebibliography}

% \end{document}
% ```
% \documentclass[11pt, a4paper]{article}
%      \usepackage[utf8]{inputenc}
%      \usepackage[T1]{fontenc}
%      \usepackage{lmodern}
%      \usepackage[french]{babel}
%      \usepackage{geometry}
%      \geometry{a4paper, margin=2.5cm}
%      \usepackage{hyperref}
     
%     \title{\textbf{Synthèse sur les Nanoparticules de BODIPY pour la Thérapie du Cancer}}
%     \author{Généré par Gemini}
%     \date{\today}
%      \begin{document}
    
%     \maketitle 
%     \section{Le Point de Départ : Le \og BODIPY de Base \fg}
%      Le \textbf{BODIPY de base} est la molécule fondamentale, un excellent colorant reconnu pour sa fluorescence intense
%       et sa grande stabilité. Cependant, pour la mission spécifique de thérapie anti-cancer, il présente des limitations
%       majeures :
    
%     \begin{itemize}
%         \item \textbf{Mauvais \og Sniper \fg (Thérapie Photodynamique - PDT) :} Il est naturellement peu efficace pour générer les espèces réactives de l'oxygène (ROS) toxiques pour les cellules cancéreuses. Il préfère libérer l'énergie lumineuse qu'il absorbe sous forme de fluorescence plutôt que de la transférer à l'oxygène pour produire des ROS.
    
%        \item \textbf{Absence de \og GPS \fg :} N'ayant pas de système de guidage, il se distribue de manière non 
%       spécifique dans l'organisme, ce qui pourrait potentiellement endommager des cellules saines.
%     \end{itemize}

% \section{L'Évolution : La Nanoparticule Thérapeutique}
%      Pour transformer ce simple colorant en un \og agent secret \fg efficace, on le modifie chimiquement pour construire une \textbf{nanoparticule de BODIPY} aux capacités améliorées.

%  \subsection{Activation du \og Sniper \fg (PDT)}
%    En \og surarmant \fg la molécule avec un atome lourd (comme l'iode), on la force à passer dans un état énergétique 
%       (l'état triplet) propice à la création de ROS. Une fois activée par la lumière, elle devient une arme chimique de précision.
    
%  \subsection{Ajout de \og l'Équipe de Démolition \fg (PTT)}
%    La nanoparticule est également conçue pour convertir l'énergie lumineuse absorbée en chaleur intense (hyperthermie),
%       provoquant la destruction thermique des cellules cibles.
    
%     \subsection{Installation du \og GPS Moléculaire \fg}
%     Pour le traitement ciblé du cancer du sein, la surface de la nanoparticule est décorée de molécules (ligands) qui
%       reconnaissent spécifiquement les récepteurs surexprimés à la surface des cellules cancéreuses mammaires. Cela
%       garantit que l'arme se concentre sur la tumeur et épargne les tissus sains.
    
%  \section{Conclusion : Une Stratégie Combinée}
%  En combinant ces améliorations, on obtient une arme théranostique (thérapie + diagnostic) puissante. Guidée par son GPS, la nanoparticule atteint sa cible, puis, sur commande (activation par la lumière), elle lance une double
%       attaque dévastatrice et localisée (PDT + PTT), tout en permettant de visualiser la zone d'action grâce à ses
%       propriétés fluorescentes intrinsèques.
   
%  \end{document}



\chapter{Méthodes computationnelles}

\section*{Revue Bibliographique Méthodes Computationnelles pour le Design Théranostique Contre le TNBC}

Le cancer du sein triple négatif (TNBC), archétype des cancers chimio-résistants du fait de l’absence des récepteurs ER, PR et HER2 \cite{Li2022,Sun2020,Demeule2021,Li2025e}, nécessite des stratégies thérapeutiques non conventionnelles, notamment les photothérapies combinées PDT/PTT \cite{He2025,Chao2023,Zhao2024}.

La chimie computationnelle, en général, et la théorie de la fonctionnelle de la densité (DFT) en particulier, est aujourd’hui indispensable pour concevoir de manière rationnelle et économique des photosensibilisateurs (PS) de nouvelle génération \cite{Ponte2022,Liu2024}.

\subsection{La Théorie de la Fonctionnelle de la Densité (DFT)}

La DFT constitue la méthode de référence pour l’étude de l’état fondamental (S$_0$) des systèmes moléculaires \cite{Ponte2022}.  
Elle permet :
\begin{itemize}
    \item l’optimisation géométrique précise des BODIPY (B3LYP-D3/def2-SVP ou def2-TZVP couramment utilisés) \cite{Ponte2022,Liu2024};
    \item le calcul des propriétés électroniques (gap HOMO–LUMO, potentiel d’oxydation/réduction) \cite{Zhao2024};
    \item la prise en compte de la solvatation biologique via le modèle CPCM(eau) \cite{Ponte2022}.
\end{itemize}

\subsection{TD-DFT : limites critiques pour les BODIPY}

Bien que largement utilisée, la TD-DFT présente des erreurs systématiques importantes sur les dérivés BODIPY à cause de leur caractère légèrement « open-shell » mild \cite{Ponte2022,Liu2024} :
\begin{itemize}
    \item Surestimation de l’énergie S$_1$ et sous-estimation de T$_1$ ;
    \item Erreur moyenne absolue (MAE) sur $\lambda_{\text{max}}$ souvent comprise entre 0,3 et 0,6~eV ;
    \item Prédiction très imprécise de l’écart singulet–triplet $\Delta E_{\text{ST}}$ (erreurs fréquemment > 0,5~eV), rendant impossible une évaluation fiable du rendement ISC et donc du potentiel PDT.
\end{itemize}


Ces limitations sont désormais bien documentées dans la littérature BODIPY et aza-BODIPY \cite{He2025,Ponte2022,Liu2024}.

\subsection{Vers la précision chimique : ADC(2) et les méthodes $\Delta$DFT / OO-DFT}

Les méthodes de référence actuelles (2023–2025) pour le design de PS BODIPY sont :

\begin{itemize}
    \item \textbf{ADC(2)/def2-TZVP ou cc-pVTZ} pour les excitations verticales et la prédiction précise de $\lambda_{\text{max}}$ (erreur typique < 0,1~eV) \cite{Ponte2022}.
    
    \item \textbf{$\Delta$SCF / $\Delta$UKS ou $\Delta$ROKS} (souvent avec $\omega$B97X-D4 ou PBE0-D4) pour les énergies adiabatiques des états S$_1$ et T$_1$, ainsi que pour l’écart $\Delta E_{\text{ST}}$ avec une précision chimique (< 0,05~eV) \cite{He2025,Zhao2024,Liu2024}.
    
    \item \textbf{FIC-NEVPT2 ou CASSCF/ZORA} pour le calcul des constantes de couplage spin–orbite (SOC) lorsque des atomes lourds (I, Br) sont présents \cite{Ponte2022}.
\end{itemize}


Ces protocoles sont aujourd’hui considérés comme le standard pour le design in silico de BODIPY NIR actifs en PDT/PTT combinée \cite{He2025,Zhao2024}.

\begin{table}[htbp]
\centering
\caption{Comparaison des méthodes computationnelles pour le design de BODIPY théranostiques (2022–2025)}
\begin{tabularx}{\linewidth}{>{\raggedright\arraybackslash}Xccc}
\toprule
Méthode & Précision $\lambda_{\text{max}}$ & Précision $\Delta E_{\text{ST}}$ & Coût computationnel \\
\midrule
TD-DFT (B3LYP, CAM-B3LYP) & ±0,3–0,6 eV & >0,5 eV (mauvais) & Faible \\
ADC(2)                   & ±0,1 eV          & –                  & Moyen \\
ΔSCF/ΔUKS–ΔROKS           & ±0,05 eV         & ±0,05 eV           & Moyen–élevé \\
FIC-NEVPT2/CASSCF     & –                & SOC précis         & Élevé \\
\bottomrule
\end{tabularx}
\source{\footnotesize Sources : \cite{Ponte2022,Liu2024,He2025,Zhao2024} (et travaux 2024–2025 non encore indexés).}
\end{table}

\subsection{Conclusion de la revue méthodologique}

Le protocole computationnel le plus robuste et le plus utilisé en 2025 pour la conception rationnelle de nanoparticules BODIPY ciblant le TNBC est donc :

\begin{enumerate}
    \item Optimisation S$_0$ → B3LYP-D3/def2-TZVP + CPCM(eau)
    \item Excitations verticales → RI-ADC(2)/def2-TZVP
    \item États relaxés S$_1$ et T$_1$ → ΔUKS ou ΔROKS (ωB97X-D4 recommandé)
    \item Couplage spin-orbite → FIC-NEVPT2 ou CASSCF/ZORA
\end{enumerate}

Ce workflow garantit une précision chimique sur $\lambda_{\text{max}}$, $\Delta E_{\text{ST}}$ et SOC, permettant de sélectionner avec confiance les candidats les meilleurs candidats BODIPY avant synthèse et tests biologiques sur lignées TNBC (MDA-MB-231, HCC1937, etc.).



\chapter{Fenêtre  Thérapeutique}

\section{Définition de la Fenêtre Thérapeutique}

La fenêtre thérapeutique, souvent appelée « fenêtre biologique » (\textit{therapeutic window} ou \textit{optical window} en anglais), est un concept fondamental en photothérapie. Elle représente la gamme de longueurs d'onde de lumière qui optimise l'interaction entre la lumière et le tissu cible tout en minimisant les interférences avec les composants biologiques endogènes \cite{Yun, Li_2020}.

%\subsection{Définition et Plages Spectrales}

\begin{itemize}
    \item La fenêtre thérapeutique se situe principalement dans la région du Proche Infrarouge (NIR) \cite{Li_2020, Wu_2020}.
    
    
    \item Cette région est privilégiée car elle présente une absorption et une diffusion (\textit{scattering}) minimales par les tissus biologiques \cite{Jacques, Yang_2015, Park_2021, Wu_2020}.
    
    \item Elle est classiquement divisée en deux sous-fenêtres principales :
   
        •  \textbf{Fenêtre Thérapeutique I (NIR-I)} : Se situe généralement entre 650--950~nm (ou parfois 600--900~nm) \cite{Yang_2015, Bai_2020, Yuan_2022, Lopes_2023, Zhao_2021}.
        
        •  \textbf{Fenêtre Thérapeutique II (NIR-II)} : S'étend de 1000--1700~nm (ou parfois 950--1700~nm) et offre une pénétration encore plus profonde \cite{Luo_2023, Ding_2014, Yang_2020, Dai_2009}. Certains travaux étendent même la définition du NIR-II jusqu'à 3000~nm \cite{Dai_2022}.
   
\end{itemize}

\section{Importance Cruciale dans la Pénétration de la Lumière}

L'efficacité de toute photothérapie( PDT, PTT ou combinée), dépend de la capacité de la lumière à atteindre la tumeur avec une intensité suffisante pour activer l'agent photoactif( photosensibilisateur), surtout dans le cas des tumeurs solides et profondes \cite{Li_2020, Han_2021}.

\subsection{Minimisation de l'Absorption Endogène}
Elle fait référence à la nécessité de choisir une longueur d'onde de lumière qui est le moins absorbée possible par les chromophores naturels (endogènes) présents dans les tissus biologiques humains \cite{Yun, Park_2021, Jacques}. Cette stratégie est fondamentale en photothérapie (PDT et PTT) car elle détermine la profondeur de pénétration de la lumière jusqu'au site tumoral ciblé \cite{Li_2020, Wu_2020}.
%(La lumière visible (par exemple, la lumière bleue ou verte) est fortement absorbée par les chromophores endogènes comme l'hémoglobine, la mélanine et l'eau \cite{Jacques, Yang_2015, Park_2021}. Dans la région NIR-I (650--950~nm), l'absorption par le sang, le collagène, la mélanine et l'eau est faible \cite{Jacques, Yang_2015, Park_2021, Wu_2020}.

\subsection{Absorption, Diffusion et Profondeur}

Les interactions de la lumière avec les tissus comprennent : la réflexion, la diffusion (\textit{scattering}) et l'absorption \cite{Kim_2020, Markolf_2019}.

\begin{itemize}
\item L'\textbf{absorption} : elle est cruciale car le photosensibilisateur (PS) au cœur de la tumeur   génére les ROS (PDT) ou la chaleur (PTT) nécessaires en fonction de la dose lumineuse reçue par l'agent absorbant \cite{Wu_2020, Li_2020}.
    \item La \textbf{diffusion} est le phénomène principal dans la région NIR et tend à devenir le mécanisme dominant à travers les tissus \cite{Markolf_2019, Yun_2017}.
    
    \item Les paramètres d'absorption et de diffusion déterminent collectivement la profondeur que la lumière peut atteindre \cite{Kennedy_2011, Park_2021}.
    \item La \textbf{pénétration profonde}: la lumière dans le NIR-I peut pénétrer profondément dans les tissus mous \cite{Jacques, Yang_2015, Park_2021}. Le NIR-II offre une pénétration encore plus profonde et un meilleur rapport signal/bruit en imagerie, grâce à une diffusion encore plus faible \cite{Luo_2023, Zhen_2021}.
\end{itemize}



%(L'utilisation du rayonnement dans la fenêtre thérapeutique du NIR est la principale manière d'augmenter le potentiel thérapeutique car elle maximise la pénétration en minimisant l'absorption et la diffusion \cite{Wu_2020, Han_2021}. Par exemple, la lumière dans le NIR-I peut pénétrer profondément dans les tissus mous \cite{Jacques, Yang_2015, Park_2021}. Le NIR-II offre une pénétration encore plus profonde et un meilleur rapport signal/bruit en imagerie, grâce à une diffusion encore plus faible \cite{Luo_2023, Zhen_2021}.)

\section{Utilisation dans la Synergie PDT/PTT}

Pour que les agents photoactifs fonctionnent, leur spectre d'absorption maximal doit coïncider avec la longueur d'onde de la lumière appliquée, qui est généralement choisie dans la fenêtre thérapeutique pour des raisons de pénétration \cite{Li_2020, Kim_2020}.

%\subsection{Accord Spectral des Agents}

Les agents PDT et PTT absorbent généralement le rayonnement aux longueurs d'onde NIR allant de 700 à 1350~nm \cite{Li_2020}. Cette gamme est compatible avec les lasers commerciaux et la fenêtre thérapeutique, ce qui permet d'atteindre les tissus plus profonds \cite{Kadkhoda_2022}.

%\subsection{Activation du Mécanisme Combiné}


     %• Les deux modalités, PDT et PTT, sont stimulées par une irradiation NIR unique \cite{Hao_2021, Sun_2022, Lee_2024}.
    
  • L'utilisation d'une lumière dans le NIR est essentielle pour garantir l'activation de la synergie au cœur de la tumeur, en particulier pour les tumeurs profondes ou les tumeurs osseuses malignes \cite{Han_2021, Deng_2021}.

    
    •  Par exemple, la PTT nécessite des agents photothermiques (PTAs) qui montrent une réponse favorable à la lumière dans la fenêtre thérapeutique I pour une pénétration optimale \cite{Jaque_2014, Vines_2019}.




\vspace{1em}


\section*{Conclusion}
L'intégration de la fenêtre thérapeutique est donc la contrainte physique primordiale de la conception : l'agent photoactif doit absorber dans cette région pour pouvoir être activé en profondeur. Dans le proche infrarouge elle constitue un élément fondamental pour  surmonter les limitations de pénétration lumineuse dans les tissus biologiques et d'activer efficacement les agents photoactifs au sein des tumeurs profondes rendant ainsi  des thérapies combinées PDT/PTT performantes.\cite{Zhao_2021}.
%(La fenêtre thérapeutique dans le proche infrarouge constitue un élément fondamental pour le développement de stratégies de photothérapie efficaces. Son exploitation optimale permet de surmonter les limitations de pénétration lumineuse dans les tissus biologiques et d'activer efficacement les agents photoactifs au sein des tumeurs profondes, ouvrant ainsi la voie à des thérapies combinées PDT/PTT performantes.)

% Bibliographie


% AFFICHAGE DES RÉFÉRENCES
\newpage

\printbibliography[title={Références bibliographiques}]
\clearpage

\end{document}
