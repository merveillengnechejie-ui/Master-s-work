\chapter{Fenêtre  Thérapeutique}

\section{Définition de la Fenêtre Thérapeutique}

La fenêtre thérapeutique, souvent appelée « fenêtre biologique » (\textit{therapeutic window} ou \textit{optical window} en anglais), est un concept fondamental en photothérapie. Elle représente la gamme de longueurs d'onde de lumière qui optimise l'interaction entre la lumière et le tissu cible tout en minimisant les interférences avec les composants biologiques endogènes \cite{Yun, Li_2020}.

%\subsection{Définition et Plages Spectrales}

\begin{itemize}
    \item La fenêtre thérapeutique se situe principalement dans la région du Proche Infrarouge (NIR) \cite{Li_2020, Wu_2020}.
    
    
    \item Cette région est privilégiée car elle présente une absorption et une diffusion (\textit{scattering}) minimales par les tissus biologiques \cite{Jacques, Yang_2015, Park_2021, Wu_2020}.
    
    \item Elle est classiquement divisée en deux sous-fenêtres principales :
   
        •  \textbf{Fenêtre Thérapeutique I (NIR-I)} : Se situe généralement entre 650--950~nm (ou parfois 600--900~nm) \cite{Yang_2015, Bai_2020, Yuan_2022, Lopes_2023, Zhao_2021}.
        
        •  \textbf{Fenêtre Thérapeutique II (NIR-II)} : S'étend de 1000--1700~nm (ou parfois 950--1700~nm) et offre une pénétration encore plus profonde \cite{Luo_2023, Ding_2014, Yang_2020, Dai_2009}. Certains travaux étendent même la définition du NIR-II jusqu'à 3000~nm \cite{Dai_2022}.
   
\end{itemize}

\section{Importance Cruciale dans la Pénétration de la Lumière}

L'efficacité de toute photothérapie( PDT, PTT ou combinée), dépend de la capacité de la lumière à atteindre la tumeur avec une intensité suffisante pour activer l'agent photoactif( photosensibilisateur), surtout dans le cas des tumeurs solides et profondes \cite{Li_2020, Han_2021}.

\subsection{Minimisation de l'Absorption Endogène}
Elle fait référence à la nécessité de choisir une longueur d'onde de lumière qui est le moins absorbée possible par les chromophores naturels (endogènes) présents dans les tissus biologiques humains \cite{Yun, Park_2021, Jacques}. Cette stratégie est fondamentale en photothérapie (PDT et PTT) car elle détermine la profondeur de pénétration de la lumière jusqu'au site tumoral ciblé \cite{Li_2020, Wu_2020}.
%(La lumière visible (par exemple, la lumière bleue ou verte) est fortement absorbée par les chromophores endogènes comme l'hémoglobine, la mélanine et l'eau \cite{Jacques, Yang_2015, Park_2021}. Dans la région NIR-I (650--950~nm), l'absorption par le sang, le collagène, la mélanine et l'eau est faible \cite{Jacques, Yang_2015, Park_2021, Wu_2020}.

\subsection{Absorption, Diffusion et Profondeur}

Les interactions de la lumière avec les tissus comprennent : la réflexion, la diffusion (\textit{scattering}) et l'absorption \cite{Kim_2020, Markolf_2019}.

\begin{itemize}
\item L'\textbf{absorption} : elle est cruciale car le photosensibilisateur (PS) au cœur de la tumeur   génére les ROS (PDT) ou la chaleur (PTT) nécessaires en fonction de la dose lumineuse reçue par l'agent absorbant \cite{Wu_2020, Li_2020}.
    \item La \textbf{diffusion} est le phénomène principal dans la région NIR et tend à devenir le mécanisme dominant à travers les tissus \cite{Markolf_2019, Yun_2017}.
    
    \item Les paramètres d'absorption et de diffusion déterminent collectivement la profondeur que la lumière peut atteindre \cite{Kennedy_2011, Park_2021}.
    \item La \textbf{pénétration profonde}: la lumière dans le NIR-I peut pénétrer profondément dans les tissus mous \cite{Jacques, Yang_2015, Park_2021}. Le NIR-II offre une pénétration encore plus profonde et un meilleur rapport signal/bruit en imagerie, grâce à une diffusion encore plus faible \cite{Luo_2023, Zhen_2021}.
\end{itemize}



%(L'utilisation du rayonnement dans la fenêtre thérapeutique du NIR est la principale manière d'augmenter le potentiel thérapeutique car elle maximise la pénétration en minimisant l'absorption et la diffusion \cite{Wu_2020, Han_2021}. Par exemple, la lumière dans le NIR-I peut pénétrer profondément dans les tissus mous \cite{Jacques, Yang_2015, Park_2021}. Le NIR-II offre une pénétration encore plus profonde et un meilleur rapport signal/bruit en imagerie, grâce à une diffusion encore plus faible \cite{Luo_2023, Zhen_2021}.)

\section{Utilisation dans la Synergie PDT/PTT}

Pour que les agents photoactifs fonctionnent, leur spectre d'absorption maximal doit coïncider avec la longueur d'onde de la lumière appliquée, qui est généralement choisie dans la fenêtre thérapeutique pour des raisons de pénétration \cite{Li_2020, Kim_2020}.

%\subsection{Accord Spectral des Agents}

Les agents PDT et PTT absorbent généralement le rayonnement aux longueurs d'onde NIR allant de 700 à 1350~nm \cite{Li_2020}. Cette gamme est compatible avec les lasers commerciaux et la fenêtre thérapeutique, ce qui permet d'atteindre les tissus plus profonds \cite{Kadkhoda_2022}.

%\subsection{Activation du Mécanisme Combiné}


     %• Les deux modalités, PDT et PTT, sont stimulées par une irradiation NIR unique \cite{Hao_2021, Sun_2022, Lee_2024}.
    
  • L'utilisation d'une lumière dans le NIR est essentielle pour garantir l'activation de la synergie au cœur de la tumeur, en particulier pour les tumeurs profondes ou les tumeurs osseuses malignes \cite{Han_2021, Deng_2021}.

    
    •  Par exemple, la PTT nécessite des agents photothermiques (PTAs) qui montrent une réponse favorable à la lumière dans la fenêtre thérapeutique I pour une pénétration optimale \cite{Jaque_2014, Vines_2019}.




\vspace{1em}


\section*{Conclusion}
L'intégration de la fenêtre thérapeutique est donc la contrainte physique primordiale de la conception : l'agent photoactif doit absorber dans cette région pour pouvoir être activé en profondeur. Dans le proche infrarouge elle constitue un élément fondamental pour  surmonter les limitations de pénétration lumineuse dans les tissus biologiques et d'activer efficacement les agents photoactifs au sein des tumeurs profondes rendant ainsi  des thérapies combinées PDT/PTT performantes.\cite{Zhao_2021}.
%(La fenêtre thérapeutique dans le proche infrarouge constitue un élément fondamental pour le développement de stratégies de photothérapie efficaces. Son exploitation optimale permet de surmonter les limitations de pénétration lumineuse dans les tissus biologiques et d'activer efficacement les agents photoactifs au sein des tumeurs profondes, ouvrant ainsi la voie à des thérapies combinées PDT/PTT performantes.)

% Bibliographie
